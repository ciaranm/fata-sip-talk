% vim: set spell spelllang=en tw=100 et sw=4 sts=4 foldmethod=marker foldmarker={{{,}}} :

\documentclass{beamer}

\usepackage{tikz}
\usepackage{xcolor}
\usepackage{complexity}
\usepackage{hyperref}
\usepackage{microtype}
\usepackage{amsmath}                   % \operatorname
\usepackage{amsfonts}                  % \mathcal
\usepackage{amssymb}                   % \nexists
\usepackage{gnuplot-lua-tikz}          % graphs
\usepackage[vlined]{algorithm2e} % algorithms

\usetikzlibrary{shapes, arrows, shadows, calc, positioning, fit}
\usetikzlibrary{decorations.pathreplacing, decorations.pathmorphing, shapes.misc}
\usetikzlibrary{tikzmark}

\colorlet{screenverylightgrey}{black!2!white}
\colorlet{screengrey}{black!30!white}

\definecolor{uofgblue}{rgb}{0, 0.321569, 0.533333}
\colorlet{uofgblue20}{uofgblue!20!white}
\colorlet{uofgblue40}{uofgblue!40!white}
\colorlet{uofgblue60}{uofgblue!60!white}
\colorlet{uofgblue80}{uofgblue!80!white}

\definecolor{uofgstone}{rgb}{0.498039, 0.454902, 0.403922}

\definecolor{uofgtdarkgreen}{rgb}{0.380392, 0.564706, 0.501961}
\definecolor{uofgtlightgreen}{rgb}{0.615686, 0.788235, 0.729412}
\definecolor{uofgtyellow}{rgb}{0.85098, 0.827451, 0.643137}
\definecolor{uofgtorange}{rgb}{0.784314, 0.694118, 0.545098}

% {{{ theme things
\useoutertheme[footline=authortitle]{miniframes}
\useinnertheme{rectangles}

\setbeamerfont{block title}{size={}}
\setbeamercolor*{structure}{fg=uofgblue}
\setbeamercolor*{palette primary}{use=structure,fg=black,bg=white}
\setbeamercolor*{palette secondary}{use=structure,fg=black,bg=uofgblue40}
\setbeamercolor*{palette tertiary}{use=structure,fg=white,bg=uofgblue}
\setbeamercolor*{palette quaternary}{fg=white,bg=black}

\setbeamercolor*{titlelike}{parent=palette primary}

\beamertemplatenavigationsymbolsempty

\tikzset{vertex/.style={draw, circle, inner sep=0pt, minimum size=0.5cm, font=\small\bfseries}}
\tikzset{notvertex/.style={vertex, color=white, text=black}}
\tikzset{plainvertex/.style={vertex}}
\tikzset{selectedvertex/.style={vertex, fill=uofgblue}}
\tikzset{vertexc1/.style={vertex, fill=uofgblue40}}
\tikzset{vertexc2/.style={vertex, fill=uofgblue}}
\tikzset{vertexc3/.style={vertex, fill=uofgtdarkgreen}}
\tikzset{vertexc4/.style={vertex, fill=uofgtorange}}
\tikzset{edge/.style={color=screengrey}}
\tikzset{bedge/.style={ultra thick}}
\tikzset{edgel1/.style={ultra thick, color=uofgblue}}
\tikzset{edgel2/.style={ultra thick, color=uofgblue40}}
\tikzset{edgel3/.style={ultra thick, color=uofgtdarkgreen}}
\tikzset{edgel4/.style={ultra thick, color=uofgtorange}}

\setbeamertemplate{title page}
{
    \vbox{}
    \begin{center}
        {\usebeamerfont{title}\inserttitle\par}
        \vskip0.5cm\par
        \begin{beamercolorbox}[sep=8pt,center]{author}
            \usebeamerfont{author}\insertauthor
        \end{beamercolorbox}
        {\usebeamercolor[fg]{titlegraphic}\inserttitlegraphic\par}
    \end{center}
    \vfill

    \begin{tikzpicture}[remember picture, overlay]
        \node at (current page.north west) {\begin{tikzpicture}[remember picture, overlay]\fill
        [fill=uofgblue, anchor=north west] (0, 0) rectangle (\paperwidth, -1.5cm);\end{tikzpicture}};
        \node [anchor=north west, shift={(0.2cm,-0.2cm)}] at (current page.north west) {\includegraphics*[keepaspectratio=true,scale=0.5]{UoG_keyline.eps}};
    \end{tikzpicture}
}

\newcommand{\frameofframes}{/}
\newcommand{\setframeofframes}[1]{\renewcommand{\frameofframes}{#1}}

\makeatletter
\setbeamertemplate{footline}
{%
    \begin{beamercolorbox}[colsep=1.5pt]{upper separation line foot}
    \end{beamercolorbox}
    \begin{beamercolorbox}[ht=2.5ex,dp=1.125ex,%
        leftskip=.3cm,rightskip=.3cm plus1fil]{author in head/foot}%
        \leavevmode{\usebeamerfont{author in head/foot}\insertshortauthor}%
        \hfill%
        {\usebeamerfont{institute in head/foot}\usebeamercolor[fg]{institute in head/foot}\insertshortinstitute}%
    \end{beamercolorbox}%
    \begin{beamercolorbox}[ht=2.5ex,dp=1.125ex,%
        leftskip=.3cm,rightskip=.3cm plus1fil]{title in head/foot}%
        {\usebeamerfont{title in head/foot}\insertshorttitle}%
        \hfill%
        {\usebeamerfont{frame number}\usebeamercolor[fg]{frame number}\insertframenumber~\frameofframes~\inserttotalframenumber}
    \end{beamercolorbox}%
    \begin{beamercolorbox}[colsep=1.5pt]{lower separation line foot}
    \end{beamercolorbox}
}

% }}}

\title[The Subgraph Isomorphism Problem: Three New Ideas]{The Subgraph Isomorphism Problem: \\ Three New Ideas}
\author[Ciaran McCreesh]{\textcolor{uofgblue}{Ciaran McCreesh} and Patrick Prosser}

\begin{document}

{
    \usebackgroundtemplate{\includegraphics*[keepaspectratio=true, height=\paperheight]{background.jpg}}
    \begin{frame}[plain,noframenumbering]
        \titlepage
    \end{frame}
}

\section{Subgraph Isomorphism}

\begin{frame}{The Subgraph Isomorphism Problem}

    \begin{itemize}
        \item Given a little \emph{pattern} graph and a large \emph{target} graph, find ``a copy
            of'' the pattern inside the target.

        \item We'll look at the \emph{non-induced} or \emph{monomorphism} variation: find an
            injective mapping that preserves adjacency, but not necessarily non-adjacency.
    \end{itemize}

\end{frame}

\begin{frame}{A CP-Like Model}

    \begin{itemize}
        \item One variable per vertex in the pattern graph. The domain is the vertex in the target
            graph that it gets mapped to.

            \begin{itemize}
                \item Reduce domains at the top of search using neighbourhood degree sequences.
            \end{itemize}

        \item For each adjacent pair of vertices in the pattern graph, their values must be adjacent
            in the target graph.

        \item All variables have different values.
    \end{itemize}

\end{frame}

\begin{frame}{Backtracking Search}

\end{frame}

\section{Supplemental Graphs}

\begin{frame}{Distance-Based Filtering}

    \begin{itemize}
        \item If two vertices are distance $d$ apart in the pattern graph, they can only be mapped
            to a pair of vertices which are within distance $d$ (or less) in the target graph.

        \item Equivalently: a monomorphism $i : P \hookrightarrow T$ is also a monomorphism $i^d
            : P^d \hookrightarrow T^d$, where $G^d$ is the graph with the same vertex set as
            $G$, and an edge between $v$ and $w$ if the distance between $v$ and $w$ in $G$ is
            at most $d$.

        \item This allows us to generate implied constraints: we're now trying to find a
            mapping $i$ which is simultaneously a monomorphism $i : P \hookrightarrow T$, and
            $i^2 : P^2 \hookrightarrow T^2$, and $i^3 : P^3 \hookrightarrow T^3$, and so on.

            \begin{itemize}
                \item So it's safe to use neighbourhood degree sequences, etc, in these graphs too.
            \end{itemize}
    \end{itemize}
\end{frame}

\begin{frame}{Path-Based Filtering}
    \begin{itemize}
        \item In practice, this only seems to be useful for $d \le 3$.

        \item Stronger: if two vertices in the pattern graph are connected by $k$ (simple) paths of
            length exactly $d$, then they can only be mapped to a pair of vertices which have at
            least $k$ paths of length exactly $d$ between them.

        \item We can view this as a graph transformation too. Let $G^{\left[d, k\right]}$ be the
            graph with the same vertex set as $G$, and an edge between $v$ and $w$ if there are at
            least $k$ paths of length exactly $d$ between $v$ and $w$ in $G$.

        \item This is \NP-hard to produce in general, but for $d \le 3$ and small $k$ we can
            calculate it quickly in practice.
    \end{itemize}
\end{frame}

\begin{frame}{Supplemental Graphs}
    \begin{itemize}
        \item We just build these graphs once, at the top of search.

        \item Other transformations are sometimes helpful too. We can either pick a good,
            general set, or use domain knowledge.

        \item Different transformations are helpful for other variations of the problem
            (induced, directed, labels, etc).
    \end{itemize}
\end{frame}

\begin{frame}{Is This Actually New?}
    \begin{itemize}
        \item SND uses distances for filtering, but not via $G^d$.
    \end{itemize}
\end{frame}

\section{Counting All-Different}

\begin{frame}{Enforcing All-Different}
    \begin{itemize}
        \item When assigning $D_v \gets w$, remove $w$ from every other domain. If a domain ends up
            being empty, fail and backtrack.

        \item This enforces the constraint, but does not provide much additional inference.
    \end{itemize}
\end{frame}

\begin{frame}{R\'egin's Matching-Based All-Different Filtering}
    \begin{itemize}
        \item Build a bipartite graph, with variables on the left, values on
            the right, and edges for allowed assignments.

        \item Find a matching that covers every variable, or fail and backtrack if there isn't
            one.

        \item Remove every edge (variable-value assignment pair) which cannot occur in any
            maximum cardinality matching.
    \end{itemize}
\end{frame}

\begin{frame}{Hall Sets}
    \begin{itemize}
        \item Why can we do this? If we have a subset of $n$ variables, whose domains include
            exactly $n$ values between them, then those values can \emph{only} be used by those
            variables.

        \item This is called a Hall set.
    \end{itemize}
\end{frame}

\begin{frame}{All-Different Filtering via Counting}
    \begin{itemize}
        \item Go through each variable, from smallest domain to largest, and take the union of
            the domains as we go along.

        \item If we reach a ``failed'' Hall set (less than $n$ values for the $n$ domains we've
            seen so far), fail.

        \item If we reach a Hall set, remove all these values from every remaining domain, reset
            the counters, and keep going.

        \item This is much faster, but may miss some deletions that matching would find.
    \end{itemize}
\end{frame}

\begin{frame}{Is This Actually New?}
    \begin{itemize}
        \item Claude-Guy Quimper and Toby Walsh used counting as preprocessing in the context of set
            variables, but they use it to determine whether it's worth trying a matching.

        \item Javier Larrosa and Gabriel Valiente counted neighbours for SIP.

        \item There are other propagators for bounds consistency.

        \item I can't find this variation in the literature, possibly because it doesn't enforce any
            particular kind of consistency.
    \end{itemize}
\end{frame}

\section{Backjumping}

\begin{frame}{Backtracking is Dumb}
    \begin{itemize}
        \item When we hit a failure, we could backtrack.
        \item Maybe the previous assignment didn't contribute to the failure, though.
    \end{itemize}
\end{frame}

\begin{frame}{Conflict-Directed Backjumping}
    \begin{itemize}
        \item Conflict-directed backjumping keeps a conflict set for each variable. We track
            which assignments removed a value from a variable. When we backtrack, if we did not
            cause the failure, we can keep going backwards.

        \item But copying conflict sets gives a performance hit inside a ``fast and dumb''
            algorithm.
    \end{itemize}
\end{frame}

\begin{frame}{Variable-Directed Backjumping}
    \begin{itemize}
        \item When we assign and fail, return which variables were involved in the failing
            constraint.

        \item When we cannot find any value to assign to a variable, return the union of the
            variables in failed sub-searches, plus ourself. (Intuition: we might be able to
            succeed, if either we had another value, or if another problematic variable had
            another value.)

        \item When a search subproblem fails, determine whether the assignment we just made
            removed any values from any of the failing variables. If not, jump back another step
            straight away.

            \begin{itemize}
                \item We don't need to track any additional information to do this, because we
                    have both the domains we were given, and the clone which has had propagation
                    applied to it.
            \end{itemize}
    \end{itemize}
\end{frame}

\begin{frame}{Backjumping plus All-Different}
    \begin{itemize}
        \item All-different(D) implies all-different(D') for any subset D' of D.

        \item If we can produce a small failed Hall set, we might be able to jump back further.

        \item We can just return the variables that we've seen so far.
            \begin{itemize}
                \item This sometimes helps a lot in practice.
                \item Maybe we could do more work to find an even better (not necessarily
                    smaller) set?
            \end{itemize}
    \end{itemize}
\end{frame}

\begin{frame}{Is This Actually New?}
    \begin{itemize}
        \item Neil Moore implemented lazy explanation generation for CP, but in a different way.

        \item Guillaume Rochart, Narendra Jussien and François Laburthe worked out better
            explanations for all-different via flows, in the context of interactive CP.
    \end{itemize}
\end{frame}

\section{Preliminary Results}

\begin{frame}{Is This Any Good?}
    \begin{itemize}
        \item Fast and dumb isn't really fashionable for CP.

        \item Backjumping isn't fashionable anywhere\ldots
    \end{itemize}
\end{frame}

\begin{frame}{The Competition}
    \begin{itemize}
        \item VF2: widely used, and extremely fast on small, sparse, low degree graphs. But if
            it doesn't find a result within ten milliseconds, it is unlikely to find a result
            within a day.

        \item LAD and SND: very clever CP-like algorithms with deep reasoning. But for some
            larger target graphs, a single propagation takes over a second.
            \begin{itemize}
                \item We do much less reasoning, but can manage $>100,000$ propagations per second.
            \end{itemize}

        \item 2063 standard benchmark instances.
    \end{itemize}
\end{frame}

\begin{frame}{Cumulative Performance}
    \input{gen-graph-cumulative}
\end{frame}

\begin{frame}{Per-Instance Comparison}
    \input{gen-graph-best-other}
\end{frame}

\begin{frame}{Is Each Feature Helpful?}
    \input{gen-graph-features}
\end{frame}

\begin{frame}{Is Backjumping Any Good?}
    \input{gen-graph-backjumping}
\end{frame}

\begin{frame}{How Much Worse is Counting All-Different?}
    \input{gen-graph-fad}
\end{frame}

\begin{frame}{Very Quick Attempt at Threaded Tree-Search}
    \only<1> {
        \begin{itemize}
            \item Half an hour's coding to check that the idea is sane.
            \item Distance 1 splitting to a queue, no extra load balancing yet.
            \item No parallelisation of supplemental graph construction yet.
            \item Speculative parallelism, so linear speedup should not be expected.
                \begin{itemize}
                    \item Not even for unsat instances, due to backjumping!
                \end{itemize}
            \item 16 threads.
        \end{itemize}
    }

    \only<2> {
        \input{gen-graph-speedup}
    }

    \only<3> {
        \input{gen-graph-parallel}
    }
\end{frame}

\section{}

\begin{frame}{What's Next?}
    \begin{itemize}
        \item Proper thread parallelism (fiddly but possible with backjumping)
        \item All the variants (labels, directed edges, induced, \ldots)
        \item Other supplemental graphs
            \begin{itemize}
                \item I can concoct additional supplemental graphs which can close half of the
                    remaining open instances, but they're rather specific to the instances.
            \end{itemize}
        \item Portfolios and instance-specific algorithm configuration?
    \end{itemize}
\end{frame}

\begin{frame}[b]
    \begin{center}
    \url{http://dcs.gla.ac.uk/~ciaran} \\
    \href{mailto:c.mccreesh.1@research.gla.ac.uk}{\nolinkurl{c.mccreesh.1@research.gla.ac.uk}}
\end{center}
\begin{tikzpicture}[remember picture, overlay]
    \node at (current page.north west) {\begin{tikzpicture}[remember picture, overlay]\fill
    [fill=uofgblue, anchor=north west] (0, 0) rectangle (\paperwidth, -1.5cm);\end{tikzpicture}};
    \node [anchor=north west, shift={(0.2cm,-0.2cm)}] at (current page.north west) {\includegraphics*[keepaspectratio=true,scale=0.5]{UoG_keyline.eps}};
\end{tikzpicture}
\end{frame}

\end{document}

