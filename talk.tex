% vim: set spell spelllang=en tw=100 et sw=4 sts=4 foldmethod=marker foldmarker={{{,}}} :

\documentclass{beamer}

\usepackage{tikz}
\usepackage{xcolor}
\usepackage{complexity}
\usepackage{hyperref}
\usepackage{microtype}
\usepackage[vlined]{algorithm2e} % algorithms

\usetikzlibrary{shapes, arrows, shadows, calc, positioning, fit}
\usetikzlibrary{decorations.pathreplacing, decorations.pathmorphing, shapes.misc}
\usetikzlibrary{tikzmark}

\colorlet{screenverylightgrey}{black!2!white}
\colorlet{screengrey}{black!30!white}

\definecolor{uofgblue}{rgb}{0, 0.321569, 0.533333}
\colorlet{uofgblue20}{uofgblue!20!white}
\colorlet{uofgblue40}{uofgblue!40!white}
\colorlet{uofgblue60}{uofgblue!60!white}
\colorlet{uofgblue80}{uofgblue!80!white}

\definecolor{uofgstone}{rgb}{0.498039, 0.454902, 0.403922}

\definecolor{uofgtdarkgreen}{rgb}{0.380392, 0.564706, 0.501961}
\definecolor{uofgtlightgreen}{rgb}{0.615686, 0.788235, 0.729412}
\definecolor{uofgtyellow}{rgb}{0.85098, 0.827451, 0.643137}
\definecolor{uofgtorange}{rgb}{0.784314, 0.694118, 0.545098}

% {{{ theme things
\useoutertheme[footline=authorinstitutetitle]{miniframes}
\useinnertheme{rectangles}

\setbeamerfont{block title}{size={}}
\setbeamercolor*{structure}{fg=uofgblue}
\setbeamercolor*{palette primary}{use=structure,fg=black,bg=white}
\setbeamercolor*{palette secondary}{use=structure,fg=black,bg=uofgblue40}
\setbeamercolor*{palette tertiary}{use=structure,fg=white,bg=uofgblue}
\setbeamercolor*{palette quaternary}{fg=white,bg=black}

\setbeamercolor*{titlelike}{parent=palette primary}

\beamertemplatenavigationsymbolsempty

\tikzset{vertex/.style={draw, circle, inner sep=0pt, minimum size=0.5cm, font=\small\bfseries}}
\tikzset{notvertex/.style={vertex, color=white, text=black}}
\tikzset{plainvertex/.style={vertex}}
\tikzset{selectedvertex/.style={vertex, fill=uofgblue}}
\tikzset{vertexc1/.style={vertex, fill=uofgblue40}}
\tikzset{vertexc2/.style={vertex, fill=uofgblue}}
\tikzset{vertexc3/.style={vertex, fill=uofgtdarkgreen}}
\tikzset{vertexc4/.style={vertex, fill=uofgtorange}}
\tikzset{edge/.style={color=screengrey}}
\tikzset{bedge/.style={ultra thick}}
\tikzset{edgel1/.style={ultra thick, color=uofgblue}}
\tikzset{edgel2/.style={ultra thick, color=uofgblue40}}
\tikzset{edgel3/.style={ultra thick, color=uofgtdarkgreen}}
\tikzset{edgel4/.style={ultra thick, color=uofgtorange}}

\setbeamertemplate{title page}
{
    \vbox{}
    \vspace*{0.5cm}
    \begin{center}
        {\usebeamerfont{title}\inserttitle\par}
        \vskip0.5cm\par
        \begin{beamercolorbox}[sep=8pt,center]{author}
            \usebeamerfont{author}\insertauthor
        \end{beamercolorbox}
        {\usebeamercolor[fg]{titlegraphic}\inserttitlegraphic\par}
    \end{center}
    \vfill

    \begin{tikzpicture}[remember picture, overlay]
        \node at (current page.north west) {\begin{tikzpicture}[remember picture, overlay]\fill
        [fill=uofgblue, anchor=north west] (0, 0) rectangle (\paperwidth, -1.5cm);\end{tikzpicture}};
        \node [anchor=north west, shift={(0.2cm,-0.2cm)}] at (current page.north west) {\includegraphics*[keepaspectratio=true,scale=0.5]{UoG_keyline.eps}};
    \end{tikzpicture}
}

% }}}

\title{The Subgraph Isomorphism Problem: \\ Three New Ideas}
\author[Ciaran McCreesh]{\textcolor{uofgblue}{Ciaran McCreesh} and Patrick Prosser}

\begin{document}

{
    \usebackgroundtemplate{\includegraphics*[keepaspectratio=true, height=\paperheight]{background.jpg}}
    \begin{frame}[plain]
        \titlepage
    \end{frame}
}

\begin{frame}{The Subgraph Isomorphism Problem}

\end{frame}

\begin{frame}{A CP-Like Model and Algorithm}

\end{frame}

\begin{frame}{Supplemental Graphs}

\end{frame}

\begin{frame}{Propagating All-Different}

    \only<1>{
        When assigning $D_v \gets v'$, remove $v'$ from every other domain. If a domain ends up
        being empty, fail and backtrack.
    }

    \only<2>{
        Assignment graph, max cardinality matching. If no matching exists, backtrack. If a matching
        exists, remove any edge (and hence variable-value pair) which cannot appear in any maximum
        cardinality matching.
    }

    \only<3>{
        Hall sets
    }

    \only<4>{
        Go through each domain, from smallest to largest, and take the union as we go along. If we
        reach a ``failed'' Hall set (less than $n$ values for the $n$ domains we've seen so far),
        fail. If we reach a Hall set, remove all these values from every remaining domain, reset the
        counters, and keep going.
    }

\end{frame}

\begin{frame}{Variable-Directed Backjumping}

    \only<1>{
        \begin{itemize}
            \item When we hit a failure, we could backtrack.
            \item Maybe the previous assignment didn't contribute to the failure, though.
        \end{itemize}
    }

    \only<2>{
        \begin{itemize}
            \item Conflict-directed backjumping keeps a conflict set for each variable. We track
                which assignments removed a value from a variable. When we backtrack, if we did not
                cause the failure, we can keep going backwards.

            \item But copying conflict sets gives a performance hit inside a ``fast and dumb''
                algorithm.
        \end{itemize}
    }

    \only<3>{
        \begin{itemize}
            \item When we assign and fail, return which variables were involved in the failing
                constraint.

            \item When we cannot find any value to assign to a variable, return the union of the
                variables in failed sub-searches, plus ourself. (Intuition: we might be able to
                succeed, if either we had another value, or if another problematic variable had
                another value.)

            \item When a search subproblem fails, determine whether the assignment we just made
                removed any values from any of the failing variables. If not, jump back another step
                straight away.

                \begin{itemize}
                    \item We don't need to track any additional information to do this, because we
                        have both the domains we were given, and the clone which has had propagation
                        applied to it.
                \end{itemize}
        \end{itemize}
    }

    \only<4>{
        \begin{itemize}
            \item AllDifferent(D) implies AllDifferent(D') for any subset D' of D.

            \item If we can produce a small failed Hall set, we might be able to jump back further.

            \item We can just return the variables that we've seen so far.
                \begin{itemize}
                    \item This sometimes helps a lot in practice.
                    \item Maybe we could do more work to find an even better (not necessarily
                        smaller) set?
                \end{itemize}
        \end{itemize}
    }

\end{frame}

\begin{frame}{Preliminary Results}

    \only<1>{
        \begin{itemize}
            \item VF2: widely used, and extremely fast on small, sparse, low degree graphs. But if
                it doesn't find a result within ten milliseconds, it is unlikely to find a result
                within a day.

            \item LAD and SND: very clever CP-like algorithms with deep reasoning. But for some
                larger target graphs, a single propagation takes over a second.

            \item We do much less reasoning, but can manage $>100,000$ propagations per second.
        \end{itemize}
    }

    \only<2>{
        Cumulative curves
    }

    \only<3>{
        Per-instance results
    }

    \only<4>{
        How good is backjumping?
    }

    \only<5>{
        How good is our weaker all-different propagation?
    }
\end{frame}

\begin{frame}{What's Next?}

    \begin{itemize}
        \item Other supplemental graphs
        \item Thread parallelism (fiddly but possible with backjumping)
        \item All the variants (labels, directed edges, induced, \ldots)
        \item Portfolios and instance-specific configuration
    \end{itemize}

\end{frame}

\begin{frame}[b]
    \begin{center}
    \url{http://dcs.gla.ac.uk/~ciaran} \\
    \href{mailto:c.mccreesh.1@research.gla.ac.uk}{\nolinkurl{c.mccreesh.1@research.gla.ac.uk}}
\end{center}
\begin{tikzpicture}[remember picture, overlay]
    \node at (current page.north west) {\begin{tikzpicture}[remember picture, overlay]\fill
    [fill=uofgblue, anchor=north west] (0, 0) rectangle (\paperwidth, -1.5cm);\end{tikzpicture}};
    \node [anchor=north west, shift={(0.2cm,-0.2cm)}] at (current page.north west) {\includegraphics*[keepaspectratio=true,scale=0.5]{UoG_keyline.eps}};
\end{tikzpicture}
\end{frame}

\end{document}

